% This is "sig-alternate.tex" V2.0 May 2012
% This file should be compiled with V2.5 of "sig-alternate.cls" May 2012
%
% This example file demonstrates the use of the 'sig-alternate.cls'
% V2.5 LaTeX2e document class file. It is for those submitting
% articles to ACM Conference Proceedings WHO DO NOT WISH TO
% STRICTLY ADHERE TO THE SIGS (PUBS-BOARD-ENDORSED) STYLE.
% The 'sig-alternate.cls' file will produce a similar-looking,
% albeit, 'tighter' paper resulting in, invariably, fewer pages.
%
% ----------------------------------------------------------------------------------------------------------------
% This .tex file (and associated .cls V2.5) produces:
%       1) The Permission Statement
%       2) The Conference (location) Info information
%       3) The Copyright Line with ACM data
%       4) NO page numbers
%
% as against the acm_proc_article-sp.cls file which
% DOES NOT produce 1) thru' 3) above.
%
% Using 'sig-alternate.cls' you have control, however, from within
% the source .tex file, over both the CopyrightYear
% (defaulted to 200X) and the ACM Copyright Data
% (defaulted to X-XXXXX-XX-X/XX/XX).
% e.g.
% \CopyrightYear{2007} will cause 2007 to appear in the copyright line.
% \crdata{0-12345-67-8/90/12} will cause 0-12345-67-8/90/12 to appear in the copyright line.
%
% ---------------------------------------------------------------------------------------------------------------
% This .tex source is an example which *does* use
% the .bib file (from which the .bbl file % is produced).
% REMEMBER HOWEVER: After having produced the .bbl file,
% and prior to final submission, you *NEED* to 'insert'
% your .bbl file into your source .tex file so as to provide
% ONE 'self-contained' source file.
%
%
% For tracking purposes - this is V2.0 - May 2012

\documentclass{sig-alternate}
\usepackage{color}
\usepackage[colorlinks,citecolor=blue]{hyperref}

\begin{document}

\conferenceinfo{Web Science}{2016 DIKU, Denmark}
\title{WS 2016 Project 1}
\numberofauthors{1} 
\author{
\alignauthor 
dpj482 Christian Edsberg Møllgaard
}
\maketitle



\section{Introduction}
In the previous study of Hansen et al.\cite{H2016}, we observed overall low error rate when predicting vaccination update using web mined and clinical data. This following study presents the same experiment, but only using web mined data from descriptions of the vaccines from ssi.dk\cite{web1} and sundhed.dk\cite{web2}. The work is done using 2 out of the 13 vaccines used in Hansen et al.\cite{H2016}. The web mines data is presented by the amount of times people Googled each word.

%It has been shown that there is a connection between certain words being searched nationally on Google and clinical data, and the amount of vaccines being sold locally\cite{H2016}. This is done by . 

%The purpose this assignment is to replicate the findings as an exercise using much simpler methods. Then compare those findings, with the real results provided with the assignment. This assignment is done using words from several texts describing the vaccine. 

I chose the vaccines MMR-2(12) and HPV-1.

The prediction is done using LASSO, and the web mined data originates from Google trends.\cite{trends}

\section{Methodology}
To accomplish this assignment I split it into several smaller pieces. 
\subsection*{Find texts}
For all vaccines I used the texts provided by the assignment text. These text may not have been the best descriptions of the vaccines, but since I have not extra knowledge of vaccines, I decided that I would not be able to find better texts. 

A big plus about these text was, that they were made using layman terms. Those would be the same words, that potential buyers would use to search for vaccines, and thus would be what were were looking for.

\subsection*{Tokenize}
Before the texts can be of any use, all the stop words has to be removed, and I should attempt to remove all words not describing the vaccine at all. 

The methods to do both of these things are provided in the assignment. First I convert all punctuation to white space. Then i split all words based on that white spaces, to get every word for it self.

Now that I only use the two text provided, I can then extract all the words from on of the files, that is included in the other file.

I remove punctuation using the character provided by pythons string library. This may unintentionally split some words that were meant to be one.

\subsection*{Querys for Google}
Now that I have each word, I get the data for each word. I then saved the result into one CSV file.

I had the problem, that some word would be downloaded in weeks instead of months which was the desired format. To handle this i made a naive split only looking at the first yeah and the first month, and combining all the weeks starting in the same month. This way the data was converted to months at the cost of some searches ending up in the wrong month. The effect is however spread out through 5 years of data, and would probably be spread out almost evenly.

\subsection*{Making the prediction}
I know nothing of machine learning, so I decided to use some kind of linear model, and hope that the result is good enough. I decided on using lasso, because the sklearn library included cross validation for lasso. The function got run on  almost default parameters. The only exception is, that I set cv to 5, to get 5 folder cross validation, and I set positive = True because it does not really make sense for coefficients to be negative. I do that because searches for one thing does not really negatively impact searches for other things.

\subsection{Naive method}To compare the result, I need another method to compare with. For this the mean of the target data is chosen. It assumes that every month, the vaccines sold is the same always, so it's not 
\section{Findings}
I compare the result of the LASSO model with the naive mean result.
My result is shown in table 1.
\begin{table}[!h]
\centering
\caption{Results}
\label{my-label}
\begin{tabular}{|l|l|l|}
\hline
vaccine   & rmse  & naive RMSE \\ \hline
HPV-1     & 10.22 & 16.28 \\ \hline
MMR-2(12) & 20.55 & 29.58 \\ \hline
\end{tabular}
\end{table}
It looks like the lasso function have done a decent job on both vaccines. 


\section{Conclusions}
There are a few problems with the target data. All target data hat zeroes in the last three months. This could be due to missing information, and the algorithm to train on missing data, and thus increase the error when there is data.


% The following two commands are all you need in the
% initial runs of your .tex file to
% produce the bibliography for the citations in your paper.
\bibliographystyle{abbrv}
\bibliography{citations}  % sigproc.bib is the name of the Bibliography in this case
% You must have a proper ".bib" file
%  and remember to run:
% latex bibtex latex latex
% to resolve all references
%
% ACM needs 'a single self-contained file'!
%
%APPENDICES are optional
%\balancecolumns
%\appendix
%Appendix A
%\balancecolumns % GM June 2007
% That's all folks!
\end{document}
